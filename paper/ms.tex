\documentclass[twocolumn,trackchanges]{aastex61}

\usepackage{amssymb,amsmath}

\newcommand{\halpha}{H$\alpha$}
\newcommand{\nh}{$N_{\rm H}$}

\received{}
\revised{}
\accepted{}
\published{}
\submitjournal{\textit{The Astrophysical Journal Letters}}

\shorttitle{} 
\shortauthors{Shimizu T. T. \emph{et\,al.}}

\begin{document}

\title{Obscuration in Seyfert 1.9 Active Galactic Nuclei}
\author[0000-0002-2125-4670]{T. Taro Shimizu}
\affiliation{Max-Planck-Institut f\"{u}r extraterrestrische Physik, Postfach 1312, 85741, Garching, Germany}
\author{Richard I. Davies}
\affiliation{Max-Planck-Institut f\"{u}r extraterrestrische Physik, Postfach 1312, 85741, Garching, Germany}
\author{BASS Collaboration}
\affiliation{Everywhere}
\correspondingauthor{T. Taro Shimizu}
\email{shimizu@mpe.mpg.de}

\begin{abstract}
\end{abstract}

\keywords{galaxies: active -- galaxies: nuclei -- galaxies: Seyfert}

\section{Introduction}\label{sec:intro}
\section{Sample and Data}\label{sec:data}
Our parent sample consists of all AGN in the BAT AGN Spectroscopic Survey (BASS; Koss et al. in preparation). BASS analysed both new and archival optical spectra for a large fraction \authorcomment1{need numbers here} of the AGN detected as part of the 70-month \textit{Swift} Burst Alert Telescope (BAT) \citep{Gehrels:2004qf, Barthelmy:2005ul} catalogue \citep{Baumgartner:2013fq}. \textit{Swift}/BAT has been continuously surveying the entire sky at high energies (14--195 keV) that allows for nearly complete samples of AGN up to the Compton thick limit ($N_{\rm H} < 10^{24}$ cm$^{-2}$) and reduces selection effects associated with host galaxy contamination and obscuration.

For this work, we need reliable measurements of the broad \halpha{} flux, intrinsic hard X-ray flux, and X-ray absorbing column density. Therefore, we chose all AGN with detected broad \halpha{} from the original BASS analysis as well as AGN that were part of the BASS X-ray spectral analysis presented in Ricci et al 2017, submitted. The key measurements obtained from the X-ray spectral analysis are \nh{} estimates and absorption corrected 14--150 keV flux (hereafter referred to as the intrinsic X-ray flux). 

Broad \halpha{} and intrinsic X-ray fluxes were converted to luminosities using either the redshift independent distances compiled from the NASA/IPAC Extragalactic Database\footnote{\url{http://ned.ipac.caltech.edu/}} or luminosity distances calculated based on the measured redshifts from the spectral analysis and our chosen cosmology ($H_{0}=70$ km s$^{-1}$ Mpc$^{-1}$, $\Omega_{m} = 0.3$).

We limited our sample to those AGN with a Seyfert type (Sy) classified according to the \citep{Winkler:1992kx} scheme. This cut resulted in 594 AGN. Of these, five sources do not have distance measurements and were immediately disregarded. 590 of the AGN have available \nh{} and intrinsic X-ray flux measurements and 297 of the AGN have a detected broad \halpha{} component. Three of the AGN with a broad \halpha{} component are detected with a significance less than 5 so we also remove these from the sample.

Our final sample contains 53, 107, 102, 100, and 227 Sy 1s, 1.2s, 1.5s, 1.9s, and 2s respectively. \authorcomment1{Maybe use a Table here to show the numbers in the full sample and then the numbers with a broad \halpha{} component?, Also a comment about the lack of Sy 1.8s?} Of these, 33, 88, 82, 79, and 6 have a significant broad \halpha{} component and. As expected based on their definition, very few Sy 2s contain a broad \halpha{} component. \authorcomment1{Need explanation for why not all Sy 1s, 1.2s, 1.5s, and 1.9s have a broad \halpha{} component.}
 
\section{Results}\label{sec:results}
\section{Discussion}\label{sec:discuss}
\section{Conclusions}\label{sec:conclude}

\acknowledgments
%\facilities{}
%\software{%
%	\textsl{astropy} \citep{Astropy:2013ek},
%	\textsl{pandas} \citep{McKinney:2010em},
%	\textsl{matplotlib} \citep{Hunter:2007},
%	\textsl{numpy} \citep{vanderWalt:2011we},
%	\textsl{scipy} \citep{Jones:2001ch}
%}
	
\bibliographystyle{aasjournal}
\bibliography{/Users/ttshimiz/Dropbox/Research/my_bib}

\end{document}